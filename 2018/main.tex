\documentclass[12pt]{article}
\usepackage{amsmath,amssymb,amsthm,amsbsy,amsfonts,mathtools}
\usepackage{url}
\usepackage{hyperref}

\usepackage{listings}
\lstdefinelanguage{julia}{
  basicstyle=\small\ttfamily,
  showspaces=false,
  showstringspaces=false,
  keywordstyle={\textbf},
  morekeywords={if,else,elseif,while,for,begin,end,quote,try,catch,return,local,abstract,function,generated,macro,ccall,finally,typealias,break,continue,type,global,module,using,import,export,const,let,bitstype,do,in,baremodule,importall,immutable},
  escapeinside={~}{~},
  morecomment=[l]{\#},
  commentstyle={},
  morestring=[b]",
}
\lstset{language=julia, numbers=left, numberstyle=\tiny, mathescape=true}

\bibliographystyle{siam}

\usepackage{todonotes}

\author{Yingbo Ma\\ \tt{yingbom@uci.edu}}
\date{March 2018}
\title{Stiffness Detection and Automatic Switching Algorithms for
OrdinaryDiffEq.jl and DAE Support for SciCompDSL.jl \\
\large{Proposal for the GSoC 2018 Project}}

\begin{document}
\maketitle
\tableofcontents

\section{Motivation \& The Project}
There are two projects that I propose for the Google Summer of Code (GSoC)
Program, one is \textit{stiffness detection and automatic switching
algorithms}, and the other is \textit{tooling of} \texttt{SciCompDSL.jl}. By
the end of the program, I hope to have implemented a production ready automatic
stiffness detection algorithms and index reduction algorithm for differential
algebraic equation (DAE).

\subsection{Goal 1: Implement stiffness detection and automatic switching
algorithms}
An ordinary differential equation (ODE) is an equality relationship among a
function $y(t)$, its independent variable $t$ and its derivatives. An $n$th
order ODE can be written as
\begin{equation}
  F(t, y(t), y'(t), \cdots, y^{(n-1)}(t)) = 0.
\end{equation}
ODEs are very important, because they are raised from every branch of science.
Most of the ODEs which people encounter have no analytical solution. Hence,
people must solve them numerically. Stiff ODEs are a kind of problem which is
very hard to solve by most explicit numerical methods, however, they can be
solved by implicit methods like the family of Rosenbrock methods or backward
differentiation formulae (BDF) without struggle. The problem is that the
methods which are good for stiff ODEs are not fast, as they have to solve for
non-linear or linear systems in each time step. Yet, most stiff problems are
not stiff throughout the whole domain of interest. This leads to the motivation
of my first project, \textit{stiffness detection and automatic switching
algorithms}. Conventionally, one uses only a single algorithm to solve for the
entire domain of interest. Thus, one is forced to use a stiff algorithm,
although there is only a very small interval in the domain that is stiff. The
auto-switching algorithm can solve this dilemma and raise the efficiency of
solving ODEs, because stiff algorithms will only be applied to the stiff part
of the domain, and non-stiff algorithms will solve the rest of the problem
quickly. This project is beneficial for most people who use the
\textit{JuliaDiffEq} ecosystem, because it removes the burden of having to test
multiple algorithms for one problem and chose the more efficient one. It can
also help to save much computation time for huge ODE systems, e.g. time
stepping of a PDE discretization.

\subsubsection{Implementation Details}

\subsection{Goal 2: Tooling for \texttt{SciCompDSL.jl}}

\subsection{Goal 3: Implement DAE index reduction algorithms}
My second project is to implement index reduction algorithms for differential
algebraic equations (DAEs). DAEs are raised naturally from Euler-Lagrange
equations with constraints and many modeling problems, and its general form is
\begin{equation}
  F(t, \vec{x}, \frac{d\vec{x}}{dt}) = \vec{0}.
\end{equation}
However, numerical DAE solvers can only handle Hessenberg index-1 and
Hessenberg index-2 forms,
\begin{align}
  &\textbf{Hessenberg index-1}
  \begin{dcases}
    y_i' &= f_i(t,y_i,z_i)\\
    0 &= g_i(t,y,z)
  \end{dcases}\\
  &\textbf{Hessenberg index-2}
  \begin{dcases}
    y'_i &= f_i(t,y_i,z_i)\\
    0 &= g_i(t,y)
  \end{dcases}
\end{align}
so one has to reduce the index of the DAE to index-1 or index-2 form. However,
it is difficult and cumbersome to reduce the indices of a DAE system by hand.
The \texttt{SciCompDSL.jl} package has enough information about a DAE system,
because it has its own computational graph of the system. This project benefits
people who work on modelings and dynamical systems when the solutions' local
coordinates are complicated and not trivial to convert to.

\subsection{Potential Hurdles}
The potential hurdles that I expect are mostly about design decisions. I
believe that I can overcome this hurdle by learning from my potential
mentors'--- Jiahao and Chris'---suggestions. For instance, when I was
implementing the order reduction algorithm for systems of ODEs, Chris taught me
that this kind of algorithms should be implemented as a transformation that is
\[
  f: \texttt{DiffEqSystem}\mapsto \texttt{DiffEqSystem},
\]
so that it fits into the design of the \texttt{SciCompDSL.jl} package, and
makes it more flexible and extendible~\cite{i49}.

Another hurdle that I expect to face during the GSoC project is that I am not
good at using \texttt{Git}. I might mess up a pull request by a bad \texttt{git
rebase}. I am confident that I can subdue this problem by being extra careful
when I merge or rebase in a \texttt{Git} repository.

\subsection{Potential Mentors}
Jiahao Chen (\texttt{@jiahao}) is going to be my primary mentor, and
Christopher Rackauckas (\texttt{@ChrisRackauckas}) is going to my secondary
mentor.

\section{Milestones}
% TODO: How will you prioritize different aspects of the project like features,
% API usability, documentation, and robustness?
There are many different aspects of a coding project like features, API
usability and documentations. I will prioritize the robustness aspect over
everything else in the beginning of the GSoC project, since the correctness is
very critical for numerical analysis softwares. I will achieve robustness by
adding extensive tests and benchmarks for all of my pull requests. I will shift
my attention to API usability after I finish all the features, as a software
must be friendly to the users to gain popularity. After I stabilize the API, I
will move my focus to documentations. I will not only write the standard
documentations in \url{http://docs.juliadiffeq.org/latest/}, but also more
tutorials in the \texttt{DiffEqTutorials.jl} repository. Here is my tentative
plan for the GSoC project.

% TODO: What milestones can you target throughout the period?

% TODO: Are there any stretch goals you can make if the main project goes
% smoothly? Tell us how you’re going to wow us with the final result!

\paragraph{Community Bonding: April 5 --- May 14}
I am going to read more literature about the stiffness of ODEs.

\section{Code Portfolio}
Here is this a list of contributions that I did during the previous GSoC
project.
\url{https://gist.github.com/YingboMa/65dbe1a43564ab367408588467ccb671}

\section{Deliverables}
% TODO: Stiffness Detection and Automatic Switching Algorithms
% TODO: SciCompDSL.jl

\section{About Me}
I am a physics major freshman in the University of California, Irvine. I am a
former GSoC student who worked on boundary value problem solvers,
Runge-Kutta-Nystr{\"o}m methods and symplectic integrators.

\subsection{Academic Details}
\begin{itemize}
  \item University: University of California, Irvine
  \item Major: Physics (Freshman)
  \item GPA: 4.0
\end{itemize}

\subsection{Contact Information}
\paragraph{Email:} mayingbo5@gmail.com
\paragraph{GitHub:} YingboMa

\section{Summer Logistics}
I expect to be able to work over 30 hours per week throughout the summer. I do
not have any summer courses nor other jobs, so I can put most of my attention
toward this GSoC project. Overall, I will be able to put 350-400 hours into
this project. If possible I plan to visit MIT Julia Lab during this GSoC
project.

\bibliography{cite}
\end{document}
