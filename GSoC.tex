% main.tex
\documentclass[a4paper,12pt,onecolumn]{article}
% figure package
\usepackage{epsfig}
\usepackage{graphicx}
\usepackage{pdfpages}
\usepackage{picture}
% math packages
\usepackage{amssymb}
\usepackage{amsmath}
\usepackage{amsthm}
\usepackage{mathrsfs}
\usepackage{enumitem} % label enumerate
\newtheorem{theorem}{Theorem}
\theoremstyle{definition}
\newtheorem{definition}{Definition}[section]
\theoremstyle{remark}
\newtheorem*{remark}{Remark}
% change Q.D.E symbol
\renewcommand\qedsymbol{$\hfill \mbox{\raggedright \rule{0.1in}{0.2in}}$}
% environment and layout packages
\usepackage[breaklinks=true,bookmarks=false,colorlinks,linkcolor=black]{hyperref}
\usepackage{framed}
\usepackage{geometry}
\geometry{left=2cm,right=2cm,top=2.5cm,bottom=3.5cm}
\usepackage{fancyhdr}
\usepackage{titlesec}
\usepackage{listings}
\lstset{%
language=Matlab,%
frame=single,%
numbers=left,%
stringstyle=\ttfamily,%
keepspaces=true,%
tabsize=4,%
basicstyle=\ttfamily
}
\usepackage{indentfirst}
\usepackage{float}
\usepackage{color}
\usepackage{xcolor}
% table packages
\usepackage{booktabs}
% font packages
% \usepackage{times}
\usepackage[LGR,T1]{fontenc}
\newcommand{\textgreek}[1]{\begingroup\fontencoding{LGR}\selectfont#1\endgroup}
% Set up paragraph
\setcounter{tocdepth}{4}
\setcounter{secnumdepth}{4}
\author{Yingbo Ma}
\title{Native Julia Solvers for Ordinary Differential Equations Boundary Value Problem: A GSoC proposal}
\begin{document}
% \nocite{*}
\maketitle
\maketitle
\tableofcontents

% \section{Synopsis} % (fold)
% \label{sec:synopsis}

% % section synopsis (end)

\section{The Project} % (fold)
\label{sec:the_project}

\subsection{Basic concepts for nonexperts} % (fold)
\label{sub:basic_concepts_for_nonexperts}
An ordinary differential equation (ODE)  is an equality relationship between a function $y(x)$
and its derivatives, and an $n$th order ODE can be written as

\[F(x, y, y', \cdots, y^{(n)}) = 0\].
%%GG also biological, economic, etc. Might be worth making this even broader. Really ODE are central to every branch of science
Physics and engineering often present ODE problems. Many physical phenomena can be reduced into an ODE. Newton's second law of motion, namely $F=ma$ can be rewritten into an ODE as $m\frac{d^2x}{dt^2}
=F(x)$, since $a$ can depend on time. ODEs can be very difficult to solve. There are many cases in which an ODE does not have an analytical solution. Therefore, numerical methods need to be used to solve
ODEs by approximating the solution. There are two kinds of ODE problems. One is initial
value problem (IVP) and the other is boundary value problem (BVP). For instance, the ODE
\[m\frac{d^2x}{dt^2} = -kx(t)\]
describes the motion for a harmonic oscillator. If the initial position $x_0$ and initial velocity
$\frac{dx_0}{dt}$ are known, then it is an IVP problem. If the condition at the ``boundary'' is known,
for instance, initial position $x_0$ and final position $x_1$, then it is a BVP problem. My
proposal is for the development of efficient and general-purpose BVP solvers for the Julia
ecosystem.
% subsection basic_concepts_for_nonexperts (end)

\subsubsection{Introduction} % (fold)
\label{ssub:introduction}
%GG I would add some more information on why, this is important for a variety of reasons etc. Let someone that is not familiar with
%   BVP solvers understand why i) this is needed for the julia ecosystem ii) why the native version is important (ie type generic etc)
The project that I propose to work on in Google's Summer of Code project is the native Julia
implementation of some BVP solving methods for ODE, namely, collocation method and shooting method.
% subsubsection introduction (end)

\subsection{Project Goals}
\subsubsection{Goal 1: Implement BVP related data structure}
A data structure to describe the BVP problem, namely, ``BVProblem''. It contains the information

\begin{align*}
	&F(x, y, y', \cdots, y^{(n)}) = 0 \\
	&\text{domin: }x \in [a, b] \\
	&\text{boundary condition: Dirichlet, Neumann, Robin}.
\end{align*}

It can be defined by

\[\text{prob} = \text{BVProblem}(f,domin,bc)\].

\begin{lstlisting}[mathescape=true]
abstract AbstractBVProblem{dType,bType,isinplace,F} <: DEProblem

type BVProblem{dType,bType,initType,F} <: AbstractBVProblem{dType,bType,F}
  f::F
  domin::dType
  bc::bType		#boundary condition
  init::initType
end

function BVProblem(f,domin,bc,init=nothing)
  BVProblem{eltype(domin),eltype(bc),eltype(init),typeof(f)}(f,domin,bc,init)
end
\end{lstlisting}

Also, I need to add a data structure for boundary conditions, which let users to define a
boundary condition easily. The following is a template.

\begin{lstlisting}[mathescape=true]
abstract AbstractBoundaryCondition

type DirichletBC <: AbstractBoundaryCondition; end
type NeumannBC <: AbstractBoundaryCondition; end
type RobinBC <: AbstractBoundaryCondition; end
\end{lstlisting}

\subsubsection{Goal 2: Implement shooting method}
The shooting method is a method that converts a BVP problem into an IVP problem and a root finding
problem. Generally, the shooting method is efficient in simple problems because it does not need
a discretization matrix. This reduces the memory overhead. The drawback is that even if the BVP problem
is well-conditioned, the root finding problem that the BVP converted to can be ill-conditioned.
Therefore, a more robust method like the collocation method is also needed, despite the fact that
the shooting shooting method is easy to implement.

This is the my current implementation of the shooting method. I will work on to generalize it later,
e.g. let user has the ability to input a ODE solver and a minimization algorithm.

\begin{lstlisting}[mathescape=true]
function solve(prob::BVProblem; OptSolver=LBFGS())
  bc = prob.bc
  u0 = bc[1]
  len = length(bc[1])
  probIt = ODEProblem(prob.f, u0, prob.domin)
  function loss(minimizer)
    probIt.u0 = minimizer
    sol = DifferentialEquations.solve(probIt)
    norm(sol[end]-bc[2])
  end
  opt = optimize(loss, u0, OptSolver)
  probIt.u0 = opt.minimizer
  @show opt.minimum
  DifferentialEquations.solve(probIt)
end
\end{lstlisting}

\subsubsection{Goal 3: Implement collocation method}
The collocation method is the idea that a solution $y(x)$ of a ODE $F(x, y, y', \cdots, y^{(n)}) = 0$
can be approximated by a linear combination of basis functions.

\[y(x) \approx \hat{y}(x) = \phi_0 + \sum_{i=1}^na_i\phi_i(x)\]

And the residual $R(x,a)$ can be written as

\[F(x, \hat{y}(x), \hat{y'}(x), \cdots, \hat{y}^{(n)}(x)) = R(x,a)\].

Collocation method forces the residual $R(x,a)$ to be $0$ for $n$ collocation points.
There are different discretization methods in collocation method, and I am going to work on
Simpson discretization, Gauss discretization, Radau discretization, and Lobatto
discretization this summer. The ``discretization'' is really a matrix $A$ that applies different
quadrature rules e.g. Simpson's rule to a vector $\vec{x}$. It forms a sparse linear system
$A\vec{x}=\vec{b}$ where $\vec{b}$ is known by the boundary condition and RHS of the ODE.

A quadrature rule is a numeric method to calculate a well-behaved define integral (no singularity).
It uses the idea that a function $f(x)$ can be represented by multiplying an orthogonal polynomial $P(x)$
and another function $W(x)$.

\[f(x) = P(x)\cdot W(x)\]

Therefore, a define integral of a function $F$ can be approximated by a linear combination of orthogonal
polynomials $p_i$ and their weights $w_i$.

\[F = \int_{a}^{b}f(x)\;dx = \int_{a}^{b}P(x)W(x) \approx \sum_{i=1}^{n}p_i(x)\cdot w_i(x)\;dx\]

\subsection{Simpson Quadrature}
Simpson's rule is a three-point Newton-Cotes quadrature rule. Its formula is
\[\int _{a}^{b}f(x)\,dx\approx {\tfrac {b-a}{6}}\left[f(a)+4f\left({\tfrac {a+b}{2}}\right)+f(b)\right].
\cite{simpson}\]

The error for the Simpson's rule is

\[{\frac{1}{90}}\left({\frac{b-a}{2}}\right)^{5}\left|f^{{(4)}}(\xi )\right|,\]

where the $\xi \in (a, b)$ by Lagrange error bound. \cite{simp}

Simpson's 3/8 rule is

\[\int _{a}^{b}f(x)\,dx\approx {\tfrac {3h}{8}}\left[f(x_{0})+3f(x_{1})+3f(x_{2})+
2f(x_{3})+3f(x_{4})+3f(x_{5})+2f(x_{6})+...+f(x_{n})\right], \cite{simp38}\]

where $h=(b-a)/h$ and $x_i = a+ih$.

\subsection{Gauss–Legendre quadrature}
The Gauss–Legendre quadrature's weight is

\[w_{i}={\frac {2}{\left(1-x_{i}^{2}\right)[P'_{n}(x_{i})]^{2}}}=
\frac{2\left(1-x_{i}^2\right)}{\left((n+1)\right)^2[P_{n+1}(x_{i})]^{2}} \cite{gauss}\]

\subsection{Lobatto Quadrature}
The Lobatto quadrature on the interval $[-1,1]$ is

\[\int _{-1}^{1}{f(x)\,dx} = \frac{2}{n(n-1)}[f(1)+f(-1)] + \sum _{i=2}^{n-1}{w_{i}f(x_{i})}+R_{n},
\cite{lobatto}\]

with the weights

\[w_{i}={\frac {2}{n(n-1)[P_{n-1}(x_{i})]^{2}}},\qquad x_{i}\neq \pm 1. \cite{lobatto}\]

$P_n$ is the $n$th order Legendre polynomial, and it can be written as

\[P_{n}(x)=2^{n} \cdot \sum _{k=0}^{n}x^{k}\binom{n}{k}\binom{{\frac {n+k-1}{2}}}{n}.\]

\subsection{Stretch Goals and Future Directions}
\subsubsection{Compatibility with the features of the common interface}
There are many features in \textit{JuliaDiffEq} common interface. They are listed in
\url{http://docs.juliadiffeq.org/latest/basics/common_solver_opts.html}, and I plan to
implement the following.
\begin{itemize}
	\item Continuous output
	\item Singularity handling
	\item \textit{Progressbars} for linear ODEs
	\item Adaptivity
\end{itemize}


\subsubsection{Weighted residual method}
Unlike the collocation method forces the residual to be zero at a finite number of
collection points, the weighted residual method minimize the residual over the entire
interval of integration.

\subsection{Timeline}
\paragraph{Pre-GSoC} % (fold)
\label{par:pre_gsoc}
%%GG this is unclear. You need to build on what it means to be working at Julia Lab. Also it will be stronger
%%   if you make this sound more positive. "I am currently working on ... this will make the transition to
%%   the GSoC project seemless as I will be able to build on the Julia/Github skillsets ..." kind of thing
Unfortunately, I have to work on function of matrix (a.k.a. matrix function) in Julia Lab.
I don't have much time to work on the GSoC project until it starts.
% paragraph pre_gsoc (end)

\paragraph{Community Bonding: May 5 (Start) - May 30} % (fold)
\label{par:community_bonding}
%GG be more specific concrete. What about git do you want to learn, what about the coding style etc
\begin{itemize}
	\item Learn more about Julia's type system and the coding style in \textit{JuliaDiffEq}.
	\item Get proficient in Git version control system.
	\item Learn more about the collocation method and optimization methods that will be
	used in the shooting method.
	\item Send a pull request.
\end{itemize}
% paragraph community_bonding (end)

\paragraph{Shooting Method: May 30 - June 15} % (fold)
\label{par:shooting_method_may_30_june_15}
\begin{itemize}
	\item Design a general framework for BVP problems. (e.g. \textit{BVProblem} and \textit{
	BoundaryCondition} data structure.)
	\item Generalize shooting method.
	\item Test shooting method with some simple problems with analytical solutions.
	\item Learn more deeply about collocation methods.
	\item Send a pull request.
\end{itemize}
% paragraph shooting_method_may_30_june_15 (end)

\paragraph{Discretization Algorithms: June 15 - July 10} % (fold)
\label{par:discretization_algorithms}
\begin{itemize}
	\item Design a basic framework for different types of discretization for solving
	BVP problems.
	\item Implement different kinds of discretization algorithms for collection methods.
	\item Optimize those discretization algorithms that are implemented with \textit{SIMD}
	and some other techniques.
	\item Send a pull request.
\end{itemize}
% paragraph discretization_algorithms (end)

\paragraph{Collocation Method: July 10 - July 31} % (fold)
\label{par:least_squares_method_july_10_july_31}
\begin{itemize}
	\item Implement the collocation method.
	\item Add more sophisticated testing BVP problems to test against the collocation method.
	\item Optimize the collocation method.
	\item Send a pull request.
\end{itemize}
% paragraph least_squares_method_july_10_july_31 (end)

\paragraph{Documentation: July 31 - August 15} % (fold)
\label{par:weighted_residual_method_july_31_august_15}
\begin{itemize}
	\item Write a documentation page for the BVP solvers that I have written.
	\item Add more tests for BVP problem and tune the algorithm.
	\item Send a pull request.
\end{itemize}
% paragraph weighted_residual_method_july_31_august_15 (end)

\paragraph{Review \& Stretch Goals: August 15 - August 29 (End)} % (fold)
\label{par:review_&_stretch_goals_august_15_august_29}
\begin{itemize}
	% \item Write a documentation page for the BVP solvers that I have written.
	\item Start to work on the stretch goals.
	\item Send a pull request.
	\item Review and test all the code that I have written in GSoC project.
\end{itemize}
% paragraph review_&_stretch_goals_august_15_august_29 (end)

\subsection{Potential Hurdles} % (fold)
\label{sub:potential_hurdles}
%%GG again build on this. Try to be more concrete in each of the hurdles and how you might be able to overcome them.
The potential hurdles I see are mostly because I have not worked in a big project like
this before, and I may use some effort to be familiar with the coding style in \textit{JuliaDiffEq}
organization. I need to be proficient with Git. I also need to be more fluent in Julia's
type system. I used to work in linear algebra which does not require much familiarity
about the software engineering side.
% subsection potential_hurdles (end)

\subsection{Mentor} % (fold)
\label{sub:mentor}
My mentor will be Christopher Rackauckas.
% subsection mentor (end)

% section the_project (end)


\subsection{Julia Coding Demo} % (fold)
\label{ssub:julia_coding_demo}
%GG Here is a selection of github repositories and other Julia projects I have contributed code towards:
Here are my code that is written in Julia.

\url{https://github.com/JuliaDiffEq/BoundaryValueDiffEq.jl}

\url{https://github.com/obiajulu/ODE.jl/tree/radau} Worked with with Joseph Obiajulu.

\url{https://github.com/YingboMa/BVP.jl}

\url{https://github.com/YingboMa/Funm.jl}
% subsection julia_coding_demo (end)

\subsection{About me} % (fold)
\label{ssub:about_me}
My name is Yingbo Ma, and I am currently a senior in Lexington Public High School. I got admitted by
University of California, Irvine (UCI). I am interested in mathematics and physics and willing to
learn new things about them. I worked with Joseph Obiajulu on \textit{ODE.jl} last summer in MIT
Julia Lab. I still go to Julia Lab regularly now, and now I am working on fixing the \textit{logm}
function in Julia base.
% subsection about_me (end)

\subsection{Contact Information} % (fold)
\label{ssub:contact_information}
\paragraph{Email:} mayingbo5@gmail.com

\paragraph{GitHub:} YingboMa

% subsection contact_information (end)

\section{Summer Logistics} % (fold)
\label{sec:summer_logistics}

%GG: What is the guidelines for how many hours you are expected from a GSoC? I would make sure that you have
%    your timeline be very similar.
\paragraph{Work hours:} I expect to be able to work over 35 hours per week though this summer.
I do not have much other thing to do besides working in this project, so I can put most of my
attention in this project. Over all I am able to put 400-500 hours into this project.

% section summer_logistics (end)

% Citations
\bibliographystyle{unsrt}
\bibliography{cite}
\end{document}
