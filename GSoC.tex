% main.tex
\documentclass[a4paper,12pt,onecolumn]{article}
% figure package
\usepackage{epsfig}
\usepackage{graphicx}
\usepackage{pdfpages}
\usepackage{picture}
% math packages
\usepackage{amssymb}
\usepackage{amsmath}
\usepackage{amsthm}
\usepackage{mathrsfs}
\usepackage{enumitem} % label enumerate
\newtheorem{theorem}{Theorem}
\theoremstyle{definition}
\newtheorem{definition}{Definition}[section]
\theoremstyle{remark}
\newtheorem*{remark}{Remark}
% change Q.D.E symbol
\renewcommand\qedsymbol{$\hfill \mbox{\raggedright \rule{0.1in}{0.2in}}$}
% environment and layout packages
\usepackage[breaklinks=true,bookmarks=false,colorlinks,linkcolor=black]{hyperref}
\usepackage{framed}
\usepackage{geometry}
\geometry{left=2cm,right=2cm,top=2.5cm,bottom=3.5cm}
\usepackage{fancyhdr}
\usepackage{titlesec}
\usepackage{listings}
\lstset{%
language=Matlab,%
frame=single,%
numbers=left,%
stringstyle=\ttfamily,%
keepspaces=true,%
tabsize=4,%
basicstyle=\ttfamily
}
\usepackage{indentfirst}
\usepackage{float}
\usepackage{color}
\usepackage{xcolor}
% table packages
\usepackage{booktabs}
% font packages
% \usepackage{times}
\usepackage[LGR,T1]{fontenc}
\newcommand{\textgreek}[1]{\begingroup\fontencoding{LGR}\selectfont#1\endgroup}
% Set up paragraph
\setcounter{tocdepth}{4}
\setcounter{secnumdepth}{4}
\author{Yingbo Ma}
\title{Native Julia Solvers for Ordinary Differential Equations Boundary Value Problem: A GSoC proposal}
\begin{document}
\maketitle
\tableofcontents

\section{Synopsis} % (fold)
\label{sec:synopsis}

% section synopsis (end)

\section{The Project} % (fold)
\label{sec:the_project}

\subsection{Basic concepts for nonexperts} % (fold)
\label{sub:basic_concepts_for_nonexperts}
An ordinary differential equation (ODE)  is an equality relationship between a function $y(x)$
and its derivatives, and an $n$th order ODE can be written as

\[F(x, y, y', \cdots, y^{(n)}) = 0\].

Physics and engineering often raise ODE problems. Many physics phenomenon can be reduced into a
ODE. Newton's second law of motion, namely $F=ma$ can rewrite into a ODE as $m\frac{d^2x}{dt^2}
=F(x)$, since $a$ can depend on time. ODEs can be really hard to solve. There are many case that
a ODE does not have an analytical solution. Therefore, numerical methods need to be used to solve
ODEs, by approximating the solution. There are two kinds of problems in ODE. One is initial
value problem (IVP) and the other is boundary value problem (BVP). For instance, the ODE
\[m\frac{d^2x}{dt^2} = -kx(t)\]
describes the motion for a harmonic oscillator. If the initial position $x_0$ and initial velocity
$\frac{dx_0}{dt}$ is known, then it is an IVP problem. If the condition at the ``boundary'' is know,
for instance, initial position $x_0$ and final position $x_1$, then it is a BVP problem. The solvers
that I am going to work on solves BVP problem.
% subsection basic_concepts_for_nonexperts (end)

\subsubsection{Introduction} % (fold)
\label{ssub:introduction}
The project that I propose to work in Google's Summer of Code project is the native Julia implementation
of some BVP solving methods for ODE, namely, collocation method and shooting method.
% subsubsection introduction (end)

\subsection{Project Goals}
\subsubsection{Goal 1: Implement BVP related data structure}
A data structure to describe the BVP problem, namely, ``BVProblem''. It contains the information

\begin{align*}
	&F(x, y, y', \cdots, y^{(n)}) = 0 \\
	&\text{domin: }x \in [a, b] \\
	&\text{boundary condition: Dirichlet, Neumann, Robin}.
\end{align*}

It can be defined by

\[\text{prob} = \text{BVProblem}(f,domin,bc)\].

\subsubsection{Goal 2: Implement shooting method}


\subsubsection{Goal 3: Implement collocation method}


\subsection{Stretch Goals and Future Directions}

\subsection{Timeline}

\subsection{Potential Hurdles} % (fold)
\label{sub:potential_hurdles}

% subsection potential_hurdles (end)

\subsection{Mentor} % (fold)
\label{sub:mentor}

% subsection mentor (end)

% section the_project (end)


\subsection{Julia Coding Demo} % (fold)
\label{ssub:julia_coding_demo}

% subsection julia_coding_demo (end)

\subsection{About me} % (fold)
\label{ssub:about_me}

% subsection about_me (end)

\subsection{Contact Information} % (fold)
\label{ssub:contact_information}

% subsection contact_information (end)

\section{Summer Logistics} % (fold)
\label{sec:summer_logistics}

% section summer_logistics (end)

% Citations
\bibliographystyle{unsrt}
\bibliography{cite}
\end{document}