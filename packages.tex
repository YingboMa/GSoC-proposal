\usepackage[margin=.75 in]{geometry}
\usepackage{physics,esint,amssymb}
\usepackage{floatrow}
\usepackage{minted}
\usepackage{graphicx}
\graphicspath{ {images/} }
\usepackage{hyperref}
\hypersetup{
    colorlinks=true,
    linkcolor=blue,
    citecolor=blue,
    filecolor=magenta,      
    urlcolor=blue,
}
 
\urlstyle{same}
\usepackage{natbib}
\usepackage[T1]{fontenc}
\usepackage{beramono}
\usepackage{listings}
\usepackage[usenames,dvipsnames]{xcolor}

%%
%% Julia definition (c) 2014 Jubobs
%%
\lstdefinelanguage{Julia}%
  {morekeywords={    exit,whos,edit,load,is,isa,isequal,typeof,tuple,ntuple,uid,hash,finalizer,convert,promote,
    subtype,typemin,typemax,realmin,realmax,sizeof,eps,promote_type,method_exists,applicable,
    invoke,dlopen,dlsym,system,error,throw,assert,new,Inf,Nan,pi,im,begin,while,for,in,return,
    break,continue,macro,quote,let,if,elseif,else,try,catch,end,bitstype,ccall,do,using,module,
    import,export,importall,baremodule,immutable,local,global,const,Bool,Int,Int8,Int16,Int32,
    Int64,Uint,Uint8,Uint16,Uint32,Uint64,Float32,Float64,Complex64,Complex128,Any,Nothing,None,
    function,type,typealias,abstract},%
   sensitive=true,%
   alsoother={$},%
   morecomment=[l]\#,%
   morecomment=[n]{\#=}{=\#},%
   morestring=[s]{"}{"},%
   morestring=[m]{'}{'},%
     morestring=[b]',
  morestring=[b]" 
}[keywords,comments,strings]%



\lstset{%
    language         = Julia,
    basicstyle       = \ttfamily,
    keywordstyle     = \bfseries\color{blue},
    stringstyle      = \color{Maroon},
    commentstyle     = \color{OliveGreen},
    showstringspaces = false,
    basicstyle=\ttfamily, 
    columns=fullflexible, % make sure to use fixed-width font, CM typewriter is NOT fixed width
    numbers=left, 
    numberstyle=\small\ttfamily\color{Gray},
    stepnumber=1,              
    numbersep=10pt, 
    numberfirstline=true, 
    numberblanklines=true, 
    tabsize=4,
    lineskip=-1.5pt,
    extendedchars=true,
    numberstyle=\small\ttfamily\color{Gray},
    breaklines=true,        
    keywordstyle=\color{Blue}\bfseries,
}

\usepackage{inconsolata} % very nice fixed-width font included with texlive-full


%\setlength{\oddsidemargin}{-.1 in}
%\setlength{\evensidemargin}{-.3 in}
%\setlength{\evensidemargin}{.0 in}
%\addtolength{\topmargin}{-1in}
%\setlength{\textwidth}{6.5in}
%\setlength{\textheight}{8in}

%
% this command enables to remove a whole part of the text 
% from the printout
% to use it just enter
% \remove{  
% before the text to be excluded and
% } 
% after the text
\newcommand{\remove}[1]{}

%
% The following macros are used to generate nice code for programs.
% See example on how to use it below
%


%%%%%%%%%%%%%%%%%%%%% program macros %%%%%%%%%%%%%%%%%
\usepackage{listings}
\usepackage{color}
 
\definecolor{codegreen}{rgb}{0,0.6,0}
\definecolor{codegray}{rgb}{0.5,0.5,0.5}
\definecolor{codepurple}{rgb}{0.58,0,0.82}
\definecolor{backcolour}{rgb}{0.95,0.95,0.92}
 
\lstdefinestyle{mystyle}{
    backgroundcolor=\color{backcolour},   
    commentstyle=\color{codegreen},
    keywordstyle=\color{magenta},
    numberstyle=\tiny\color{codegray},
    stringstyle=\color{codepurple},
    basicstyle=\footnotesize,
    breakatwhitespace=false,         
    breaklines=true,                 
    captionpos=b,                    
    keepspaces=true,                 
    numbers=left,                    
    numbersep=5pt,                  
    showspaces=false,                
    showstringspaces=false,
    showtabs=false,                  
    tabsize=2
}


\lstset{style=mystyle}


%\newcommand{\Do}{{\small\bf do}\ }
%\newcommand{\Return}{{\small\bf return\ }}
%\newcommand{\Proc}[1]{#1\+}
%\newcommand{\Returns}{{\small\bf returns}}
%\newcommand{\Procbegin}{{\small\bf begin}}
%\newcommand{\If}{{\small\bf if}\ \=\+}
%\newcommand{\Then}{{\small\bf then}\ \=\+}
%\newcommand{\Else}{\<{\small\bf else}\ \>}
%\newcommand{\Elseif}{\<{\small\bf elseif}\ }
%\newcommand{\Endif}{\<{\small\bf end\ if\ }\-\-}
%\newcommand{\Endproc}[1]{{\small\bf end} #1\-}
%\newcommand{\For}{{\small\bf for}\ \=\+}
%\newcommand{\Endfor}{{\small\bf end\ for}\ \-}
%\newcommand{\Loop}{{\bf loop}\ \= \+}
%\newcommand{\Endloop}{{\bf end\ loop}\ \-}
%\newenvironment{program}{
%    \begin{minipage}{\textwidth}
%    \begin{tabbing}
%    \ \ \ \ \=\kill
%  }{
%    \end{tabbing}
%    \end{minipage}
%  }

% a blank line
\def\blankline{\hbox{}}

%%%%%%%%%%%%%%%%%%%%% End of PROGRAM macros %%%%%%%%%%%%%%%%%


%
% The following macro is used to generate the header.
%
%

\newcommand{\lecture}[5]{
   %\pagestyle{headings}
   \thispagestyle{plain}
   \newpage
  %\setcounter{chapter}{#1}
   \setcounter{page}{#2}
%  \set\thechapter{#3}
   \noindent
   \begin{center}
   \framebox{
      \vbox{
    \hbox to 6.28in { {\bf MAT 425: Analysis III
                        \hfill Spring Semester, 2016} }
       \vspace{4mm}
       \hbox to 6.28in { {\Large \hfill Problem Set : #1 #3  \hfill} }
       \vspace{2mm}
       \hbox to 6.28in { {\it Instructor: #4 \hfill Student: #5} }
      }
   }
   \end{center}
   \markboth{Lecture #1: #3}{Lecture #1: #3}
   \vspace*{4mm}
}

%
% Use these macros for organizing sections of your notes.
% Each command takes two arguments: (1) the title of the section and and
% (2) a keyword for that section to appear in the index.  (See examples.)
% Please don't use \section, \subsection, and \subsubsection directly!
%

\newcommand{\topic}[2]{\section{#1} \index{#2} \markright{#1}}
\newcommand{\subtopic}[2]{\subsection{#1} \index{#2}}
\newcommand{\subsubtopic}[2]{\subsubsection{#1} \index{#2}}
 
%
% Convention for citations is first author's last name followed by other
% authors' last initials, followed by the year.  For example, to cite the
% seventh entry in the course bibliography, you would type: \citet{BurnsL80}
% (To avoid bibliography problems, for now we redefine the \cite command.)
%
     
\renewcommand{\cite}[1]{[#1]}


%
% Use these for theorems, lemmas, proofs, etc.
%


%\newtheorem{theorem}{Theorem}[chapter]
\newtheorem{theorem}{Theorem}
\newtheorem{conjecture}[theorem]{Conjecture}
\newtheorem{problem}[theorem]{Problem}

\newtheorem{lemma}[theorem]{Lemma}
\newtheorem{claim}[theorem]{Claim}
\newtheorem{corollary}[theorem]{Corollary}
\newcommand{\qed}{\hfill $\Box$}
\newenvironment{proof}{\par{\bf Proof:}}{\qed \par}
%\newenvironment{proof}{{\em Proof:}}{\hfill\rule{2mm}{2mm}}
\newtheorem{definition}[theorem]{Definition}

%
% Use the following for definitions.
% \bigdef is for definitions to be set off by themselves; \smalldef is for
% definitions given in the middle of a paragraph.
%
\newenvironment{dfn}{{\vspace*{1ex} \noindent \bf Definition }}{\vspace*{1ex}}
\newcommand{\bigdef}[2]{\index{#1}\begin{dfn} {\rm #2} \end{dfn}}
\newcommand{\smalldef}[1]{\index{#1} {\em #1}}
%%%%%%%%%%%%%%%%%%%%% math macros %%%%%%%%%%%%%%%%%
\usepackage{amsmath}
\usepackage{amssymb}
\usepackage{mathtools}
\DeclarePairedDelimiter\ceil{\lceil}{\rceil}
\DeclarePairedDelimiter\floor{\lfloor}{\rfloor}

\newcommand{\ep}{\epsilon}
\newcommand{\al}{\alpha}
\newcommand{\R}{\mathbb{R}}
\newcommand{\Rd}{{\mathbb{R}^{d}}}
\newcommand{\Z}{\mathbb{Z}}
\newcommand{\Zp}{\mathbb{Z}_+}
\newcommand{\C}{\mathbb{C}}
\newcommand{\Q}{\mathbb{Q}}
\newcommand{\D}{\mathbb{D}}
\newcommand{\Os}{\mathcal{O}}
\newcommand{\Hs}{\mathcal{H}}
\newcommand{\N}{\mathbb{N}}
\newcommand{\Lone}{{L^1(\R^d)}}
\newcommand{\Ltwo}{{L^2(\R^d)}}

\newcommand{\ltwo}{{\ell^2(\Z)}}

\newcommand{\fund}[1]{\pi(#1,x_0)}
\newcommand{\0}{\mathbf{0}}
\newcommand{\me}[2][E]{m_{*}({#2})}
\newcommand{\Nm}{\mathcal{N}}
\newcommand{\can}{\mathcal{C}}

\newcommand*\BF[1]{\mathbf{#1}}
\newcommand*\dif{\mathop{}\!\mathrm{d}}

\def\keywords{\vspace{.5em}
{\textit{Keywords}:\,\relax%
}}
\def\endkeywords{\par}